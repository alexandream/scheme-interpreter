\section{Macros}
\label{sec:macros}

Macros são o que torna possível que grande parte das estruturas sintáticas 
necessárias em um interpretador \textit{Scheme} seja implementada em código
\textit{Scheme}. Além disso, o mesmo mecanismo pode ser utilizado por um 
programador final para criar suas próprias estruturas sintáticas, linguagens
específicas de domínio ou para otimizar cálculos possíveis de serem executados
em tempo de compilação.

Uma implementação do relatório \acs{R7RS} deve conter um sistema bastante
avançado de macros, chamado \texttt{syntax-rules}, que permite que as macros
sejam definidas em termos de reconhecimento de padrões e substituição, além de
garantir que as macros são ``higiênicas'', ou seja, que os símbolos criados
durante a transformação de uma macro não conflitem com símbolos do contexto em
que a macro foi instanciada e que vinculações inseridas por uma transformação 
de macro não capturem inadvertidamente utilizações de variáveis do código
usado como base para a transformação.

Devido aos problemas no compilador descritos na seção
\ref{ss:solucoes-compilador}, no entanto, não é possível implementar um sistema
de macros higiênicas -- e o sistema de macros baseado em reconhecimento de
padrões seria inútil, visto que para resolver possíveis problemas de higiene o
programador deve, manualmente, interferir na substituição de alguns símbolos. 

Foi implementado então o mecanismo de macros procedurais utilizado pelo outro
dialeto Lisp em amplo uso atualmente: \textit{ANSI Common Lisp}. Um mecanismo
de macros procedurais permite que um programador defina funções que são
aplicadas sobre o código dado como entrada para uma macro e estas funções são
então responsáveis por transformar a entrada e gerar a saída desejada.

Para minimizar problemas de higiene, uma operação chamada \texttt{(gensym)}
pode ser utilizada para gerar um símbolo garantido de não conflitar com outro
símbolo qualquer no mesmo programa que um programador deve judiciosamente
utilizar quando há possibilidade de um conflito de símbolos. Na figura 
\ref{fig:exemplo-defmacro1} pode-se verificar um exemplo simples, porém com
erros bastante complicados de uma macro procedural.

\begin{figure}[h!]
\begin{lstlisting}
      (define-rewriter or2
        (lambda (expr)
          (let ((op1 (car expr))
                (op2 (cadr expr)))
          
		    `(let ((temp ,op1))
		       (if temp temp ,op2)))))
\end{lstlisting}
\caption{Exemplo incorreto de uma macro implementando a operação \texttt{or}}
\label{fig:exemplo-defmacro1}
\end{figure}

Esta macro, embora pareça simples, possui alguns grandes problemas. O mais
óbvio aparece com o vínculo introduzido na variável \texttt{temp}: este pode
conflitar com o código contido tanto em \texttt{op1} como em \texttt{op2}, e a
expressão teria um significado completamente diferente da intenção do
programador. O problema aqui é que para ter certeza que uma utilização de macro
vai funcionar, o programador final tem de saber quais símbolos são ``seguros''
de se usar em conjunto com a macro.

A princípio pode-se modificar sutilmente esta definição de macro, utilizando a
operação \texttt{(gensym)} para retirar esta dependência com o símbolo
\texttt{temp}, como visto na figura \ref{fig:exemplo-defmacro2}. No entanto,
tudo pode dar errado se o programador final manipular o escopo global -- se as
definições de \texttt{let} ou \texttt{if} forem alteradas, esta macro não tem
mais o mesmo significado que tinha quando foi definida. Estes problemas são
solucionados por um mecanismo de macros higiênicas, como o descrito na seção
\ref{ss:solucoes-macros}.

\begin{figure}[h!]

\begin{lstlisting}
      (define-rewriter or2
        (lambda (expr)
          (let ((op1 (car expr))
                (op2 (cadr expr))
                (sym-temp (gensym)))
          
			`(let ((,sym-temp ,op1))
               (if ,sym-temp ,sym-temp ,op2)))))
\end{lstlisting}
\caption{Outro exemplo incorreto de uma macro implementando a operação
         \texttt{or}} 
\label{fig:exemplo-defmacro2} \end{figure}

O sistema de macros é composto por dois elementos principais: uma lista de 
vínculos que associa cada um dos símbolos marcados como macros à função
\textit{Scheme} pré-compilada que deve ser utilizada para efetuar a
transformação da entrada e uma função utilizada internamente para identificar
aplicações de macros e chamar a função de transformação necessária em cada caso
identificando e aplicando transformações numa dada entrada até que esta não 
possa mais ser interpretada como uma aplicação de macro.

Para efetuar a chamada destas funções de transformação, o sistema de macros
conta com uma instância própria da máquina virtual, com um contexto limitado de
variáveis (contendo os mesmos vínculos que o contexto de variáveis globais),
que é utilizado para interpretar o código das funções de transformação quando
necessário.

A lista de vínculos é alimentada pela instrução \sctt{bind-macro} da máquina
virtual, que teve de ser implementada para fazer com que, mesmo ante a
incapacidade do compilador de fornecer informações de contexto estático, um
sistema de macros pudesse ser implementado. É importante notar que, uma vez que
um símbolo é associado a uma macro, este não será mais visto como variável por
parte alguma do código, pois o mecanismo de macros usa uma lista de vínculos
própria para identificar que símbolos estão associados a macros e esta lista
não é intermediada por nenhum tipo de informação sobre contexto estático
durante a compilação. Esta é uma das deficiências do compilador descritas na
seção \ref{ss:solucoes-compilador}.

\subsection{Estendendo o sistema de macros}
\label{ss:solucoes-macros}

A partir do momento que esta implementação é estendida com as mudanças indicadas
na seção \ref{ss:solucoes-compilador}, torna-se possível implementar um 
mecanismo de macros higiênicas conhecido como ``macros baseadas em renomeação
explícita`` que, embora ainda seja um mecanismo procedural de macros, foi
desenvolvido visando ser utilizado como base para uma implementação do sistema
\texttt{syntax-rules} descrito no \acs{R7RS}\cite{scheme-of-things}.

Uma das modificações indicadas na seção \ref{ss:solucoes-compilador} dizia que
o compilador deveria ser capaz de dizer, dado um símbolo e um contexto estático,
qual a distância entre o contexto estático dado e o contexto estático mais
próximo no qual o símbolo possui um vínculo. Esta informação é crucial para
a implementação de macros por renomeação explícita.

Para que esta informação seja útil, as seguintes mudanças precisam ser 
realizadas, na máquina virtual e no compilador: 

\begin{itemize}

\item a máquina virtual deve passar a suportar instruções \sctt{lookup} que não
só informam o símbolo indicando a variável a ser carregada, mas também uma
distância indicando quantos contextos de escopo devem ser caminhados até se
procurar por este símbolo;

\item o compilador deve passar a compilar todas as referências a variáveis
como instruções \sctt{lookup} que usam tanto o símbolo da variável quanto a
distância até o contexto mais próximo em que se sabe que ela foi definida.

\end{itemize}

Em grande parte dos casos esta mudança não altera em nada a semântica de
referência a variáveis durante o tempo de execução, embora a deixe ligeiramente
mais rápida: a referência a uma variável não precisa mais procurar em cada
contexto se a variável tem um valor definido neste contexto até encontrar, pode
navegar diretamente ao contexto especificado e obter o valor.

No entanto, agora que a máquina virtual compreende estas novas instruções
\sctt{lookup}, elas podem ser utilizadas para renomear símbolos, como descrito
por Jonathan Rees\cite{scheme-of-things} para implementar macros higiênicas.

Para isso o compilador deve ser capaz de reconhecer
não apenas símbolos mas também uma estrutura chamada \textit{símbolos
renomeados} na hora de realizar uma referência a um valor. Um símbolo renomeado
representa uma referência a uma variável ou operador sintático baseado em um
contexto específico, independente do contexto atual.

A estrutura de um símbolo renomeado é de um par contedo um símbolo comum (que é
o mesmo que o símbolo original antes de ser renomeado) e uma referência ao
contexto utilizado para renomear este símbolo. A operação de compilar uma
referência a um símbolo renomeado \sctt{s}, composto do símbolo \sctt{x} e do
contexto \sctt{e}, pode ser descrita como: 

\begin{enumerate}

\item encontre a distância \sctt{d1} entre \sctt{e} e o contexto mais próximo
de \sctt{e} em que \sctt{x} foi vinculado;
 
\item encontre a distância \sctt{d2} entre o contexto atual e \sctt{e};

\item a referência para \sctt{s} pode ser compilada como uma instrução 
\sctt{lookup} estendida como descrito acima, usando o símbolo \sctt{x} e a 
distância \sctt{d1} + \sctt{d2}. 

\end{enumerate}

Agora que o compilador possui a capacidade de compreender tanto referências
simples a símbolos como referências a símbolos renomeados, é possível definir
um mecanismo de macros procedurais baseadas em renomeação explícita:

Uma definição de macro cria um objeto macro composto de dois elementos: Uma
função de transformação, que recebe como parâmetros a expressão a ser
transformada e o contexto estático no qual esta macro foi definida. O
compilador então, após criar o objeto macro, associa um nome a este objeto no
contexto estático relevante.

Ao encontrar a aplicação de uma macro, ou seja, identificar uma forma
aplicativa do tipo \texttt{(\textsc{identificador} parametros ...)} em um
contexto estático em que \sctt{identificador} está associado a um objeto macro,
o compilador cria uma função \texttt{renomear}, de domínio um símbolo \sctt{s}
e imagem a um símbolo renomeado, que retorna o símbolo renomeado criado a
partir do símbolo \sctt{s} e do contexto estático \sctt{e} em que a macro foi
criada, se este símbolo puder ser encontrado em \sctt{e}.  Caso \sctt{s} não
possa ser encontrado em \sctt{e}, \texttt{renomear} retorna um novo símbolo
codificado de forma a não poder conflitar com nenhum outro símbolo escolhido
pelo programador -- por exemplo, utilizando o mecanismo \texttt{(gensym)}
mencionado anteriormente.

De posse desta função \texttt{renomear} o compilador então interpreta a
execução da função de transformação da macro instanciada passando como
parâmetros a expressão passada como os parâmetros da aplicação da macro e a
função \texttt{renomear} descrita acima. Cabe ao programador que criou a 
definição da macro, então, utilizar a função \texttt{renomear}  em todos os 
símbolos nos quais se deseja que a higiene seja preservada -- o que, em geral,
significa todos os símbolos utilizados na substituição.

Tendo estas ferramentas à disposição, vamos re-escrever a macro \texttt{or2}
descrita anteriormente de forma correta. No código da figura
\ref{fig:renomeacao-explicita} \texttt{define-er-syntax} é o identificador
utilizado para indicar ao compilador que deve-se definir uma macro utilizando a
estratégia de renomeação explícita.

\begin{figure}[h!]
\begin{lstlisting}
     (define-er-rewriter or2
       (lambda (expressao rename)
         (let ((primeiro (cadr expressao))
               (segundo  (caddr expressao))
               (r-let    (rename 'let))
               (r-if     (rename 'if))
               (r-temp   (rename 'temp)))
               
           `(,r-let ((,r-temp ,primeiro))
              (,r-if ,r-temp 
                ,r-temp
                ,segundo)))))
\end{lstlisting}
\caption{Macro definida com o mecanismo de renomeação explícita.}
\label{fig:renomeacao-explicita}
\end{figure}

Pode-se observar que, em cada situação em que antes havia um problema possível
por quebra de higiene na implementação desta macro, foi substituído o símbolo
em questão pelo resultado da renomeação deste símbolo no contexto de definição
da macro. Por exemplo, a aplicação de \texttt{(rename 'if)} garante que o valor
retornado é um símbolo renomeado que referencia o valor do símbolo \texttt{if}
como este estava definido no momento em que a macro foi criada: levando-se em
consideração a execução de \texttt{define-er-syntax} no nível superior, este
valor é o operador sintático pré-definido de controle de fluxo condicional.

Tendo o sistema de macros higiênicas por renomeação explícita funcionando, o que falta
para uma implementação completa do sistema de macros como definido no \acs{R7RS}
é uma implementação do mecanismo de reconhecimento de padrões e substituição
conhecido como \texttt{syntax-rules}. A implementação do mesmo baseado no
sistema de macros por renomeação explícitas, no entanto, é relativamente simples
e uma implementação em apenas 250 linhas de código \texttt{Scheme} é fornecida
por Jonathan Rees no mesmo \textit{paper} em que detalha o mecanismo de macros
por renomeação explícita descrita acima\cite{scheme-of-things}.
