A intenção deste trabalho foi demonstrar que a linguagem \textit{Scheme}, como
descrita pelo relatório \acs{R7RS} em seu terceiro \textit{draft}, pode ser
implementada de forma simples sem a necessidade de conhecimento muito
aprofundado sobre implementação de interpretadores de linguagens de
programação. Inicialmente, a linguagem e o contexto histórico necessário para
entender esta necessidade foram expostos.

Através do desenvolvimento de uma implementação e da pesquisa e escolha das
técnicas mais simples encontradas, foi possível concluir uma implementação de
uma parcela significativa da linguagem descrita pelo \acs{R7RS}. Também foi
possível  listar diversas técnicas, suficientemente simples, que podem ser
utilizadas para realizar esta implementação -- mesmo que o programador utilize
como base apenas descrições das técnicas específicas na literatura, sem uma
profunda pesquisa das áreas tocadas por cada uma das técnicas utilizadas.

Nos pontos em que a implementação não conseguiu cumprir com todos os requisitos
de uma implementação da linguagem \textit{Scheme}, foram apresentadas técnicas
que podem ser utilizadas para circundar problemas de projeto desta implementação.
Estas técnicas, embora não implementadas, aparentam ser simples o suficiente para
não impossibilitar a implementação ou demandar um esforço de pesquisa muito maior
que o esperado pelas premissas deste trabalho.

A maior dificuldade encontrada no decorrer deste trabalho foi a enorme pesquisa
necessária para encontrar técnicas suficientemente simples que ainda assim 
pudessem ser usadas em conjunto para realizar a implementação da maior parcela
possível da linguagem \textit{Scheme}. Desta dificuldade e de uma escolha ruim
em uma das técnicas utilizadas derivam a maior parte dos problemas e 
incompatibilidades desta implementação com os requisitos iniciais.

Outra dificuldade foi a falta de experiência do autor com a implementação de
linguagens de programação, em especial na área de representação de dados tipados
em tempo de execução e gerência automática de memória. No entanto, a conclusão
de uma implementação tão próxima dos requisitos iniciais, não obstante a falta
de experiência, foi um dos indicadores mais fortes para que, no ponto de vista do
autor, este experimento possa ser visto como um sucesso em demonstrar que a
linguagem \textit{Scheme} cumpre com a premissa que este trabalho se propôs a
verificar.

A execução deste trabalho, sua implementação e documentação foram,
indubitavelmente, o elemento do curso de graduação em ciência da computação que
mais contribuiu para meu crescimento e realização acadêmica e pessoal. Todas as decisões e
toda a pesquisa necessária para fazer com que este projeto se tornasse uma
realidade foram suficientes para despertar um novo interesse sobre as áreas de
implementação de linguagens de programação, máquinas virtuais e gerenciadores
automáticos de memória -- e mostrar novos caminhos a partir dos quais continuar
o aprendizado sobre estas áreas.

\section{Trabalhos Futuros}
\label{sec:trabalhos-futuros}

Devido ao limite de tempo disponível para o desenvolvimento deste projeto, não foi
possível corrigir as falhas encontradas durante a implementação, embora fossem
conceitualmente simples. Trabalhos futuros poderão implementar as sugestões
descritas nas seções \ref{ss:solucoes-compilador} e \ref{ss:solucoes-macros} e
aproximar mais a implementação atual do padrão definido pela linguagem.

O próprio relatório \acs{R7RS} está ainda em fase de ratificação, de forma que
um projeto futuro pode adaptar esta implementação às mudanças nos \textit{drafts}
subsequentes e na versão final quando esta for finalmente ratificada.

Outra oportunidade para um projeto futuro é o de estender esta implementação
para completar os módulos opcionais e implementar as funções e tipos que foram
deixados em aberto por este trabalho.

Uma sugestão para projeto futuro que não se liga diretamente à implementação,
mas às técnicas e aos resultados obtidos, é compilar as técnicas utilizadas em
um material mais fluido e sequencial, com descrições mais didáticas, para
compor um programa de uma disciplina introdutória sobre a implementação de
interpretadores de linguagens de programação para um curso de graduação em
ciência da computação.

Todas as técnicas utilizadas podem ser estendidas e alteradas para prover
melhor desempenho, menor utilização de memória e melhor detecção de informação
de erros.  Em especial, esta implementação poderia ser adaptada para gerar
código para uma máquina virtual existente, como a ``Parrot VM'' e colher as
otimizações constantes feitas nesta.
