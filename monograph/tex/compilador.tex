\section{Compilador}
\label{sec:compilador}

A estratégia utilizada para o compilador é derivada de um compilador utilizado
como exemplo por Kent Dybvig em sua tese de doutorado\cite{3imp}. Este compilador,
embora simples, sofre de algumas deficiências que serão discutidas na seção
\ref{ss:solucoes-compilador}.

Durante a etapa de compilação uma Expressão-S representando código Scheme é
traduzida para uma Expressão-S representando a implementação deste código nas
instruções utilizadas pela máquina virtual. É importante observar que, como a compilação se
dá em termos de Expressões-S como entrada, nada impede que um usuário forneça
ao compilador uma Expressão-S que não é diretamente derivada da representação
textual do programa. Esta característica é essencial para o funcionamento do
mecanismo de tradução de macros.

Neste ponto o funcionamento do mecanismo de macros ainda não será discutido
profundamente. No entanto, é necessário deixar claro que, antes de a compilação
propriamente dita de uma expressão ter início, o compilador invoca o sistema de
macros para pré-processar a Expressão-S recebida como entrada. Este
pré-processamento é necessário para transformar a entrada em um conjunto de
Expressões-S composta apenas das poucas operações primitivas sobre as quais o
compilador tem conhecimento. Isto simplifica muito a etapa de compilação,
permitindo ao compilador se preocupar apenas com um número reduzido das
estruturas sintáticas definidas no R7RS-small como \textit{formas sintáticas
primitivas}.

A estratégia de implementação utilizada é a mais ingênua possível. De fato,
esta implementação não mantém informações de análise estática relacionadas a
escopo, o que a faz incapaz de representar o mecanismo de macros definido pela
linguagem Scheme. Esta implementação conta com um mecanismo alternativo de
macros procedurais, compatível com os mecanismos tradicionais de outras
linguagens da família Lisp, que é discutido na seção \ref{sec:macros}. Uma
descrição das mudanças necessárias para esta implementação seja capaz de 
suportar um conjunto maior da linguagem \textit{Scheme}
vista na seção \ref{ss:solucoes-compilador}.

Apenas dez tipos de expressão são reconhecidos pelo compilador: Referência a variável,
constantes,  expressões literais, criação de procedimentos, condicionais,
atribuições, definições, definições de macro, obtenção de continuação e
aplicação de função. A partir destes tipos primitivos de expressão, com a ajuda
do mecanismo de macros, todas as outras estruturas sintáticas da linguagem são
construídas, simplificando significativamente o projeto do compilador.

O processo de compilação inicia-se com dois parâmetros: a expressão a ser
compilada e a próxima etapa da computação que está sendo efetuada. Quando
estamos realizando a compilação de uma expressão no escopo superior como no
ambiente interativo, esta próxima etapa normalmente é um pedaço de código pré
compilado correspondendo à instrução \sctt{halt} da máquina virtual. Esta
expressão representando a próxima etapa da computação pode então ser utilizada
para compor o parâmetro das instruções da máquina virtual que indica qual a
próxima instrução a ser executada.

O compilador então invoca o sistema de macros de forma que a expressão dada como
entrada seja reduzida para uma expressão primitiva, caso seja uma invocação de
uma forma sintática derivada.

Com uma expressão primitiva o compilador pode então verificar cada um dos casos
possíveis e gerar o código necessário. 

Durante a discussão das regras utilizadas
pelo compilador para gerar código, a notação utilizada é a mesma de código
\textit{Scheme}. A expressividade da linguagem permite que, para efeitos de 
discussão transformações razoavelmente complexas sejam 


O compilador utiliza as seguintes regras para a geração de código, dada uma 
expressão de entrada e um parâmetro indicando o código seguinte a ser executado
pela expressão sendo compilada:

\begin{itemize}

\item Caso a expressão seja um símbolo, é retornada uma instrução \sctt{lookup} 
recebendo como parâmetros o símbolo em questão e o código a ser executado em
seguida. Este processo pode ser visto na figura \ref{fig:compile-lookup}.

\begin{figure}
\begin{lstlisting}
COMPILAR-REFERENCIA(Referencia, Proximo)
    RESULTADO := MAKE-LOOKUP(Referencia.Variavel, Proximo)
\end{lstlisting}
\caption{Processo de compilação de uma referência a variável}
\label{fig:compile-lookup}
\end{figure}

\item Caso a expressão seja uma lista o compilador verifica o primeiro elemento
desta. Se este for um símbolo, e for um dos símbolos indicadores de formas
sintáticas primitivas (\texttt{if}, \texttt{quote}, \texttt{define},
\texttt{lambda}, \texttt{set!}, \texttt{define-rewriter} ou \texttt{call/cc}) o 
compilador então executa uma das transformações a seguir, de acordo com o 
símbolo encontrado:
\begin{itemize}
 
 \item Caso o símbolo encontrado seja \texttt{if}, primeiro são compilados
 ambos os ramos da operação (a alternativa é assumida como \#F caso não seja
 informada pelo usuário) recebendo como código a ser executado em seguida o
 mesmo código que esta compilação de \texttt{if} recebeu. 
 
 Então uma instrução \sctt{test} é gerada recebendo como parâmetros os códigos
 compilados de cada um dos ramos da operação compilados acima. 
 
 Finalmente é
 retornado o resultado da compilação da condição do \texttt{if} usando como
 código seguinte a instrução \sctt{test} gerada anteriormente. Este processo
 pode ser visto na figura \ref{fig:compile-test}.  
 
\begin{figure}
\begin{lstlisting}
COMPILAR-CONDICIONAL(Condicional, Proximo)
    Consequencia := COMPILAR(Condicional.Consequencia, Proximo)
    Alternativa := COMPILAR(Condicional.Alternativa, Proximo)
    Test := MAKE-TEST(Consequencia, Alternativa)
    Resultado := COMPILAR(Condicional.Condicao, Test)
\end{lstlisting}
\caption{Processo de compilação de uma expressão condicional}
\label{fig:compile-test}
\end{figure}

 \item Caso o símbolo encontrado seja \texttt{quote}, o compilador retorna uma
 instrução \sctt{constant} que possui como parâmetros o primeiro parâmetro 
 passado para a operação quote como valor da constante e o código a ser
 executado em seguida.
 
 \item Caso o símbolo encontrado seja \texttt{define}, primeiro o compilador
 cria uma instrução \sctt{bind} que tem como parâmetros o símbolo recebido como
 primeiro parâmetro da expressão \texttt{define} e o código recebido para
 executar em seguida, então a expressão recebida como segundo parâmetro da
 expressão \texttt{define} é compilada usando a instrução \sctt{bind} criada
 como código a ser executado em seguida. Este processo pode ser visto na figura
 \ref{fig:compile-define}
 
\begin{figure}
\begin{lstlisting}
COMPILAR-DEFINICAO(Definicao, Proximo)
    Bind := MAKE-BIND(Definicao.Variavel, Proximo)
    Resultado := COMPILAR(Definicao.Expressao, Bind)
\end{lstlisting}
\caption{Processo de compilação de uma definição de variável}
\label{fig:compile-define}
\end{figure}

 \item Caso o símbolo encontrado seja \texttt{set!}, o comportamento é
 praticamente idêntico ao anterior para \texttt{define} exceto por ser utilizada
 uma instrução \sctt{assign} no lugar da instrução \sctt{bind}. Este processo
 pode ser visto na figura \ref{fig:compile-assign}
 
\begin{figure}
\begin{lstlisting}
COMPILAR-ATRIBUICAO(Atribuicao, Proximo)
    Bind := MAKE-ASSIGN(Atribuicao.Variavel, Proximo)
    Resultado := COMPILAR(Atribuicao.Expressao, Bind)
\end{lstlisting}
\caption{Processo de compilação de uma atribuição}
\label{fig:compile-assign}
\end{figure}

 \item Caso o símbolo encontrado seja \texttt{define-rewriter}, novamente,
 o comportamento é análogo ao de \texttt{define} exceto por ser utilizada uma
 instrução \sctt{bind-macro} no lugar da instrução \sctt{bind}.
 
 \item Caso o símbolo encontrado seja \texttt{call/cc}, primeiro o compilador 
 cria uma instrução \sctt{apply} e compila a expressão recebida como único 
 parâmetro de \texttt{call/cc} usando esta instrução \sctt{apply} como código
 seguinte. 
 
 Então uma instrução \sctt{argument} é criada usando a expressão compilada 
 anteriormente como código seguinte e uma instrução \sctt{save} é criada usando
 a instrução \sctt{argument} criada como código seguinte.
 
 Finalmente, se o código recebido no início da compilação como código a ser
 executado em seguida for uma instrução \sctt{return}, ou seja, isto é parte de 
 uma chamada terminal, a instrução \sctt{save} criada anteriormente é retornada.
 Caso contrário, uma instrução \sctt{frame} é criada usando o código recebido
 no início da compilação e a instrução \sctt{save} criada como parâmetros, e esta
 instrução \sctt{frame} é retornada. Este processo pode ser visto com mais 
 detalhes na figura \ref{fig:compile-save}.

\begin{figure}
\begin{lstlisting}
COMPILAR-CALL/CC(Call/CC, Proximo)
    Apply := MAKE-APPLY()
    Expressao := COMPILAR(Call/CC.Expressao, Apply)
    Argument := MAKE-ARGUMENT(Expressao)
    Save := MAKE-SAVE(Argument)

    IF IS-RETURN(Proximo) THEN
        Resultado := Save
    ELSE
        Resultado := MAKE-FRAME(Proximo, Save)
\end{lstlisting}
\caption{Processo de compilação de uma criação de continuação.}
\label{fig:compile-save}
\end{figure}

 \item Caso o símbolo encontrado seja \texttt{lambda}, primeiro uma instrução
 \sctt{return} é criada. O compilador então compila o corpo da função a ser
 criada usando a instrução \sctt{return} como código a ser executado em seguida.
 
 Finalmente uma instrução \sctt{closure} é criada, usando a lista de argumentos e
 o corpo compilado da função como parâmetros, e retornada. Este processo pode ser
 visto na figura \ref{fig:compile-lambda}.

\begin{figure}
\begin{lstlisting}
COMPILAR-LAMBDA(Lambda, Proximo)
    Return := MAKE-RETURN()
    Conteudo := COMPILAR(Lambda.Corpo, Return)
    Resultado := MAKE-CLOSURE(Lambda.Argumentos, Conteudo, Proximo)
\end{lstlisting}
\caption{Processo de compilação de uma expressão \textit{lambda}}
\label{fig:compile-lambda}
\end{figure}
\end{itemize}

\item Caso a expressão seja uma lista que não se enquadre em nenhum dos casos
anteriores, então assume-se que seja uma aplicação de função. Primeiro uma
instrução \sctt{apply} é criada e o primeiro elemento da lista (que corresponde
à qual função deve ser chamada) é compilado usando como código seguinte esta
instrução \sctt{apply}. 

Então, para cada um dos elementos seguintes da lista, que serão os argumentos
da aplicação de função, uma instrução \sctt{argument}  é criada usando como
código seguinte ou o código compilado da função gerado anteriormente 
(apenas para o primeiro argumento) ou o código gerado pela compilação do
argumento anterior na lista (para todos os demais) e o argumento atual é
compilado usando como código seguinte esta instrução \sctt{argument} criada.

Finalmente, caso o código recebido inicialmente nesta compilação como código
seguinte seja uma instrução \sctt{return} (detecção de chamadas terminais),
o código gerado para o último argumento é retornado. Caso contrário, uma nova
instrução \sctt{frame} é criada, usando como parâmetro o código recebido como
seguinte e usando como código seguinte o resultado da compilação do último 
argumento, e esta instrução \sctt{frame} é retornada. Este processo pode ser
visto com mais detalhes na figura \ref{fig:compile-application}.

\begin{figure}
\begin{lstlisting}
COMPILAR-APLICACAO(Aplicacao, Proximo)
    Apply := MAKE-APPLY()
    Operador := COMPILAR(Aplicacao.Operador, Apply)
    Acumulado := Operador
    
    FOREACH Operando in Aplicacao.Operandos DO
        Argument := MAKE-ARGUMENT(Acumulado)
        Acumulado := COMPILAR(Operando, Argument)
    
    IF IS-RETURN(Proximo) THEN
        Resultado := Acumulado
    ELSE
        Resultado := MAKE-FRAME(Proximo, Acumulado)
\end{lstlisting}
\caption{Processo de compilação de uma expressão de aplicação de função}
\label{fig:compile-application}
\end{figure}
\item Caso a expressão não se enquadre em nenhum dos casos anteriores, 
assume-se que seja uma constante, então uma instrução \sctt{constant} é criada
usando a expressão como parâmetro e o código recebido como seguinte é usado 
para executar em seqüência.

\end{itemize}

Com estes passos completa-se todo o trabalho que o compilador deve realizar,
ficando o resto da computação a cargo da máquina virtual. A criação de todo o
código gerado para a máquina virtual pelo compilador utiliza o sistema de 
gerência de memória, sendo que elementos intermediários são salvos na pilha de 
proteção de memória para evitar que estes sejam inadvertidamente coletados caso
uma coleta seja invocada durante a execução do compilador. 

\subsection{Estendendo o compilador}
\label{ss:solucoes-compilador}

Embora o compilador apresentado seja simples de implementar,
ele possui alguns problemas que criam empecilhos para toda a implementação.
Durante a implementação de outras funcionalidades notou-se que a estratégia
utilizada seria insuficiente para representar toda a linguagem descrita pelo
\acs{R7RS}-small. Embora este problema tenha sido notado, já era tarde demais
para alterar a base do sistema de compilação sem arriscar atrasar todo o resto
do projeto.

Nesta seção se discute a principal falha de projeto do compilador e uma
estratégia que pode ser utilizada para estender a implementação atual de forma
que esta seja capaz de atingir um maior grau de compatibilidade com a linguagem
descrita pelo \acs{R7RS}-small.

A falha principal existente no compilador é a falta de manutenção de um
contexto estático de nomes e escopo. Esta falha cria os seguintes problemas:

\begin{enumerate}[I]

 \item Os nomes dos operadores sintáticos pré-definidos da linguagem não podem
ser redefinidos pelo usuário como variáveis em um escopo local.

 \item Os nomes das macros presentes no sistema ou definidas pelo usuário, após
serem definidos, não podem ser redefinidos localmente como variáveis.

 \item Se torna impossível implementar macros locais (uma conseqüência direta
do problema 2 descrito acima).
 
 \item Se torna impossível implementar macros higiênicas.

\end{enumerate}


O fato de o compilador não manter este contexto estático de nomes e escopo o
torna incapaz de analisar de forma mais sofisticada o código que recebe.
Isto criaa necessidade de soluções pouco elegantes na máquina virtual (como a
vinculação de macros em tempo de execução) e impossibilitando qualquer análise
que deva ser realizada em tempo de compilação. A seguir é
discutida uma evolução do compilador que o torna capaz de resolver os problemas
I, II e III. O problema IV requer um pouco mais de sofisticação e será
discutido na seção \ref{ss:solucoes-macros}.

Para a solucionar estes problemas, o compilador deve ser capaz de tratar de
forma especial as construções primitivas que introduzem novos vínculos de nomes
e manter uma tabela de símbolos que o possa responder, durante a compilação,
as seguintes perguntas dado um símbolo qualquer \sctt{x} e um
contexto qualquer \sctt{e}:

\begin{enumerate}
 \item \sctt{x} representa um operador sintático pré-definido no contexto \sctt{e}? 

 \item Caso \sctt{x} represente um operador sintático no contexto \sctt{e},
qual é este operador?

 \item \sctt{x} está associado a uma macro no contexto \sctt{e}? 
 
 \item Caso \sctt{x} esteja associado a uma macro no contexto \sctt{e}, qual a
função de transformação de \sctt{x} e em qual contexto \sctt{x} foi definido?

 \item \sctt{x} representa uma variável comum no contexto \sctt{e}? 

 \item Caso \sctt{x} represente uma variável no contexto \sctt{e}, qual a
distância entre \sctt{e} e o contexto em que \sctt{x} foi vinculado mais
recentemente. Onde a distância entre dois contextos \sctt{e1} e \sctt{e2}
definida como a diferença de altura entre dois nós em uma árvore de contextos.

\end{enumerate}

Dado que o compilador pode responder às perguntas 1 a 5 acima, é possível
restringir o escopo dos operadores sintáticos (tanto pré-definidos, como
macros) durante um bloco qualquer vinculando localmente o nome associado a um
operador sintático a uma variável, como visto na figura \ref{lst:compiler-scope}
abaixo, resolvendo de uma só vez os problemas I, II e III listados acima.

\begin{figure}[h!]
\begin{lstlisting}
    (define (minha-func x y)
      ;; Aqui o valor de "if" e' o valor do operador sintatico primitivo.
      (if (> x 10)
        (let ((if (+ x y)))
           ;; Aqui o valor de "if" e' a soma dos parametros x e y
           (* if if))
        (* x y)))
\end{lstlisting}
\caption{Exemplo de utilização de nomes pré definidos como variáveis}
\label{lst:compiler-scope}
\end{figure}


Como o compilador precisa conhecer apenas as formas primitivas, deixando as
formas derivadas a cargo de transformações de macro que ocorrem antes da
compilação propriamente dita de um determinado trecho de código, é possível
introduzir a inteligência necessária para o compilador responder às perguntas
acima no tratamento específico de cada caso que influencie a manutenção do
contexto estático. A estratégia delineada a seguir exemplifica como estender o
compilador de forma a responder às perguntas 1 a 5:


\begin{enumerate}[(A)]

\item Ao iniciar, o compilador possui um contexto estático em que os operadores
sintáticos pré-definidos pela linguagem, as macros pré-definidas pela
biblioteca padrão e as variáveis e funções pré-definidas pela biblioteca padrão
estão associadas com seus respectivos nomes aos valores esperados para cada
caso.

\item Sempre que o compilador encontrar uma operação \texttt{define}, altere o contexto
estático atual de modo que o valor do símbolo definido aponte para uma variável
comum.

\item Sempre que o compilador encontrar uma operação \texttt{lambda}, compile o código
do corpo desta operação em um contexto estático derivado do contexto atual tal
que os parâmetros formais da expressão \texttt{lambda} estejam associados a variáveis
comuns.

\item Sempre que o compilador encontrar uma operação \texttt{define-syntax}, altere o
contexto estático atual de modo que o valor do símbolo definido aponte para o
objeto macro resultante de compilar a definição de macro fornecida.

\item Sempre que o compilador encontrar uma operação \texttt{let-syntax}, compile o
código do corpo desta operação em um contexto estático derivado do contexto
atual tal que os valores dos símbolos definidos nesta operação estejam
associados aos objetos macros resultantes de compilar as suas respectivas
definições de macros.

\end{enumerate}

Dadas estas regras acima, o compilador mantém um contexto estático atualizado
ao compilar cada fragmento do programa, de forma a sempre ser capaz de
responder às perguntas 1 a 5 enumeradas anteriormente. Utilizando a estratégia
de derivar um contexto estático de outro utilizando encadeamento de contextos,
a resposta para a pergunta 6 segue trivialmente: é apenas necessário navegar
para cima na cadeia de contextos até encontrar o contexto mais próximo no qual
a variável pedida foi definida, contando o número de elos percorridos.

Com esse conjunto pequeno de mudanças, então, grande parte das maiores falhas
desta implementação em relação ao descrito pelo \acs{R7RS}-small são
resolvidas.  Em especial o problema quanto à utilização de palavras-chave como
nomes de variáveis já é resolvido diretamente simplesmente pela manutenção dos
contextos estáticos.
