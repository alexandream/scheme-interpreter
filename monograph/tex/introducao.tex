\section{Motivação e objetivos}
\label{sec:motivacao}

Este projeto apresenta uma implementação da mais recente edição do relatório
reconhecido como \textit{de facto standard} da linguagem de programação
\textit{Scheme}, o \acs{R7RS}. O objetivo é demonstrar a aderência ao
princípio, definido no plano de ratificação do relatório, que requer que a
implementação da linguagem seja passível de ser realizada como meio de pesquisa
e aprendizado sobre linguagens de programação sem a necessidade de técnicas
demasiado avançadas\cite{wg1-charter}.

Scheme é uma linguagem de programação da família Lisp, que difere dos outros
dialetos Lisp  mais conhecidos (como \textit{Common Lisp} e \textit{Emacs
Lisp}) por utilizar uma estratégia de escopo totalmente léxico, possuir
continuações como objetos de primeira classe, e compartilhar o espaço de nomes
entre funções e variáveis. Também se destaca por manter um considerável foco em
minimalismo, preferindo primitivas poderosas a complexas funcionalidades
especializadas\cite{first-report}. Estas primitivas são, em geral, altamente ortogonais e podem
ser utilizadas para criar novas expressões compostas,

Por causa destas características, \textit{Scheme} tornou-se ao longo do tempo
uma plataforma bastante utilizada para o ensino de computação e conceitos de
linguagens de programação. Em especial, a facilidade de implementar novos
conceitos e funcionalidades sobre a linguagem base pré-definida fez com que
Scheme fosse bastante utilizado no estudo exploratório de novos conceitos de
linguagens de programação. Esta linguagem é utilizada em diversos cursos
iniciais de computação e programação, inclusive em cursos de nível
médio\cite{sicp}\cite{htdp}\cite{scheme-schools}. Também é frequentemente utilizada como linguagem de exemplo na
literatura da área de projeto de linguagens de programação e design de
interpretadores\cite{plai}\cite{eopl}.


A utilização constante de \textit{Scheme} dentro da comunidade acadêmica
contribuiu para que os comitês responsáveis pelas evoluções da linguagem, até a
ratificação do \acs{R5RS} em 1998, dessem grande importância à possibilidade de
a linguagem ser implementada de forma simples como objeto de estudo e
experimentação. No entanto, os objetivos por trás da ratificação do \acs{R6RS}
em 2007 estavam mais alinhados com as necessidades do mercado de
desenvolvimento de aplicações e padronização entre implementações, o que
distanciou a linguagem significativamente dos objetivos mantidos até o
\acs{R5RS}. Este distanciamento, junto com algumas incompatibilidades entre a
linguagem descrita pelo \acs{R6RS} e a descrita pelo \acs{R5RS}, fez com que o
\acs{R6RS} fosse ignorado por uma parcela significativa da comunidade de 
implementadores\cite{r6rs-controversy}.

Esta divisão levou a comunidade a um esforço de unificação dos objetivos que
é encarnado na iminente ratificação do \acs{R7RS}. Para evitar os
problemas anteriores foi decidido que a evolução da linguagem se daria em duas
frentes: uma linguagem ``pequena'' (posteriormente nomeada simplesmente
\acs{R7RS}-small), guiada pelos princípios de minimalismo que historicamente
foram seguidos, e uma linguagem ``grande'' (\acs{R7RS}-large), baseada na
R7RS-small com a intenção de levá-la em direção aos objetivos do \acs{R6RS}
sem tanta oposição\cite{r7rs-steering-committee-position}. 

O processo de ratificação da \acs{R7RS}-small está próximo do fim, com o terceiro
draft tendo sido publicado em março de 2011\cite{r7rs-draft3}. Entretanto, a linguagem
\acs{R7RS}-large foi deixada para definição futura pois todos os
participantes do grupo de trabalho dedicado à \acs{R7RS}-large também fazem
parte do grupo de trabalho dedicado à \acs{R7RS}-small e preferiram terminar o
trabalho de base antes de iniciar o trabalho mais completo nas camadas acima.

O histórico da baixa aceitação do \acs{R6RS} demonstra que uma linguagem que
rompa muito com as necessidades dos grupos interessados em utilizar
\textit{Scheme} como uma linguagem de pesquisa e ensino tem pouca probabilidade
de ser amplamente implementada pela comunidade.

Inserido neste contexto, este trabalho é uma tentativa de demonstrar a
aderência aos princípios de minimalismo e capacidade de implementação de
\textit{Scheme} como objeto de estudo de alunos de graduação em ciência da
computação. Este objetivo será atingido por meio da apresentação de uma
implementação  da linguagem como descrita pelo terceiro \textit{draft} do
\acs{R7RS}, utilizando técnicas simples e diretas passíveis de serem abordadas
por um aluno de graduação.

Deixando de lado os diversos pequenos módulos opcionais descritos no relatório,
este trabalho se concentra no módulo ``\texttt{base}'', obrigatório para uma
implementação \textit{Scheme}.


\section{Trabalhos relacionados}
\label{sec:trabalhos_relacionados}

Até o momento da conclusão deste trabalho, apenas duas outras tentativas de
implementar o conteúdo dos drafts da \acs{R7RS}-small foram encontradas:
\textit{chibi-scheme}, criada por Alex Shinn, presidente do grupo de trabalho
responsável pelo \acs{R7RS}-small\cite{chibi}; e \textit{r7rs-bridge} criada por
Okumura Yuki\cite{r7rs-bridge}.

Os objetivos destas, no entanto, diferem significativamente deste trabalho:

\subsection{Chibi Schemee}
\label{sub:chibi_scheme}

Chibi Scheme é uma tentativa de implementar a linguagem \acs{R7RS}-small (além de
alguns dialetos mais antigos como o \acs{R5RS}) como uma biblioteca de extensão e
linguagem de script para programas em C. Seu foco principal está em diminuir o
tamanho do executável final e aumentar a performance. Embora o código seja
relativamente bem comentado, não há qualquer intenção de escolher
funcionalidades e estratégias pensando na simplicidade de implementação em
detrimento dos objetivos citados acima. Desta forma, Chibi Scheme não é uma
implementação viável para demonstração da capacidade de se implementar a
linguagem \acs{R7RS}-small como objeto de estudo.


\subsection{\acs{R7RS} Bridge}
\label{sub:r7rs_bridge}

R7RS-Bridge, de fato, é uma tentativa completamente diferente, focada em
desenvolver uma linguagem compatível com \acs{R7RS}-small através de
bibliotecas para sistemas Scheme \acs{R6RS}. Portanto, este pré-requisito em
ter uma implementação anterior do \acs{R6RS} (o relatório controverso que o
\acs{R7RS} tenta substituir) claramente coloca as intenções de
\acs{R7RS}-Bridge distantes de uma implementação de \acs{R7RS}-small como objeto de
estudo.

\section{Estrutura da monografia}
\label{sec:estrutura_da_monografia}

O capítulo \ref{cap:scheme} descreve a linguagem \textit{Scheme} e suas
funcionalidades mais notáveis, que a diferenciam da maior parte das linguagens
de uso geral em ampla utilização atualmente.

O capítulo \ref{cap:estrategia} descreve as estratégias utilizadas na
implementação deste trabalho, e está dividido nas seções descritas as seguir:

\begin{itemize}

\item A seção \ref{sec:memoria} descreve a estrutura interna utilizada para
representar valores no interpretador, bem como as estratégias utilizadas para
reutilização da memória alocada anteriormente para objetos que passaram a ser
impossíveis de serem utilizados pelo programa;

\item A seção \ref{sec:leitor} descreve o leitor, uma fase tradicional da
compilação de programas Lisp que corresponde à análise léxica e parte da
análise sintática;

\item A seção \ref{sec:compilador} descreve o compilador, que implementa o
restante da análise sintática e gera código para uma máquina virtual simples;

\item A seção \ref{sec:maquina-virtual} descreve a arquitetura da máquina
virtual implementada, seus registradores e instruções;


\end{itemize}

O capítulo \ref{cap:resultados} lista os resultados conseguidos com a
implementação, identifica que partes da especificação foram implementados e
aponta as partes em que esta implementação falha em seguir a especificação.

O capítulo \ref{cap:conclusao} detalha os objetivos atingidos com este trabalho
e lista possíveis trabalhos futuros baseados neste.

