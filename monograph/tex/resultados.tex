\section{Visão geral}
\label{sec:introducao-resultados}

O principal resultado atingido com esta implementação foi um ambiente interativo
\textit{Scheme} capaz de reconhecer a maior parte da linguagem descrita pelo 
terceiro \textit{draft} do relatório \acs{R7RS}-small. Em especial, foram
implementados:

\begin{itemize}

\item todos os conceitos definidos na seção 3 do relatório, ``\textit{Basic
Concepts}'';

\item todos os tipos de expressão primitivos definidos pela seção 4.1 do
relatório, ``\textit{Primitive expression types}'';

\item 15 dos 24 tipos de expressão derivados definidos pela seção 4.2 do
relatório, ''\textit{Derived expression types}'', sendo que dos 9 não
implementados 5 (\texttt{cond}, \texttt{case}, \texttt{letrec*}, \texttt{do},
\texttt{case-lambda}) são formas sintáticas trivialmente implementáveis
diretamente como macros \textit{Scheme} utilizando as expressões básicas já
disponíveis;

\item aproximadamente 60\% (8 de 13) dos tipos primitivos definidos pelo \acs{R7RS}, sendo
 os outros 40\% de implementação opcional ou análoga a de algum tipo já
implementado (como, por exemplo, vetores que compartilhariam muito código com
a implementação dada para strings);

\item e aproximadamente 75\% (133 de 179) dos símbolos descritos no módulo
\texttt{base} do \acs{R7RS}, sendo que dos 46 que faltam, 31 são acessores ou
construtores de tipos não implementados.

\end{itemize}

No entanto, algumas funcionalidades notáveis não foram implementadas, ou não foram
implementadas de acordo com a especificação, estas são:

\begin{itemize}

\item todo o mecanismo de módulos -- embora o interpretador seja capaz de carregar 
código de arquivos externos, todos os identificadores definidos por estes são 
injetados diretamente no escopo global;

\item a linguagem de reconhecimento de padrões utilizada para o sistema de macros
baseadas em regras;

\item a possibilidade de utilizar palavras-chave da linguagem (como \texttt{if} ou
\texttt{lambda}) como nome de variáveis;

\item as macros geradas pelo sistema de macros procedural não são higiênicas;

\end{itemize}

Todos estes problemas derivam de um mesmo problema na estratégia de construção do 
compilador: o compilador utilizado não mantém conhecimento suficiente sobre o 
contexto estático para poder dar as informações necessárias para que as diferenças
de escopo sejam gerenciadas caso a caso.

Este problema, no entanto, foi detectado em uma fase demasiado avançada do
desenvolvimento em que a sua solução teria grande probabilidade de comprometer
a finalização do resto da implementação e, por isso, resolveu-se continuar por
utilizar a estratégia atual. Na seção \ref{ss:solucoes-compilador} é realizada
uma discussão de como o compilador poderia ser modificado para resolver este
problema base, enquanto que na seção \ref{ss:solucoes-macros} se discute como
o sistema de macros higiênicas poderia ser criado a partir do momento que o
compilador fosse reestruturado.

Mesmo com essas falhas, a implementação base deste trabalho pode ser vista como
uma implementação próxima o suficiente da linguagem definida pelo
\acs{R7RS}-small e utiliza técnicas suficientemente simples para ser de
perfeito entendimento por um aluno de graduação em ciência da computação. 

A implementação total se dá em 4359 linhas de código \textit{C++} e
\textit{Scheme}, das quais a implementação do núcleo do interpretador consome
apenas 1448. Todo o restante do trabalho de implementação é relacionado à
implementação dos tipos primitivos e biblioteca padrão em cima do núcleo
original.

Analisando apenas o tamanho do esforço de implementação, parece
razoável supor que a implementação de um subconjunto significativo desta
implementação possa ser realizada por pequenos grupos de alunos de graduação
como trabalho de conclusão de disciplinas na área de implementação de
interpretadores, dado que as técnicas envolvidas sejam abordadas durante o curso.

Analisando a complexidade das técnicas utilizadas dentro das respectivas áreas
necessárias para a implementação de um interpretador, nada causa motivo para
supor que esta implementação seja tecnicamente acima do desempenho esperado de
um aluno de graduação ao fim de um curso. Deve-se observar que as técnicas
escolhidas figuram quase sempre entre as técnicas introdutórias na literatura
para cada um dos tópicos abordados\cite{jones:gc}\cite{plai}. A exceção é a
utilização da máquina \sctt{secd} como máquina virtual já que a utilização de
uma máquina virtual para um interpretador raramente é abordada pela literatura
introdutória na área.

No entanto, a literatura deixa para discutir tópicos como continuações como
objetos de primeira classe e eliminação de chamadas terminais quando a base
dos interpretadores de exemplo já é complexa o suficiente para comportá-los
sem a necessidade de uma máquina virtual. Utilizando a estrutura proposta por
Dybvig\cite{3imp} é possível realizar a implementação da máquina virtual e o 
compilador que a alimenta de forma suficientemente simples e, ao mesmo tempo,
resolver os tópicos complexos que a literatura normalmente evita em textos
introdutórios. Na literatura sobre máquinas virtuais, no entanto, a máquina
\sctt{secd} ganha novamente o aspecto de um elemento introdutório utilizado
para familiarizar o leitor com o assunto, como visto no livro ``Virtual Machines''
de Iain Craig\cite{craig:vm}.


\section{Alguns problemas}
\label{sec:problemas-resultados}

Embora a implementação como um todo seja simples, esta simplicidade levou a
alguns problemas de implementação que poderiam ser corrigidos com o uso de
algumas técnicas mais avançadas. A seguir são discutidos brevemente alguns
destes problemas.

\subsection{O Compilador}
\label{ss:compilador-resultados}

Como já discutido, o compilador apresentado nesta implementação é incapaz
de reconhecer toda a linguagem \textit{Scheme} como definida pelo \acs{R7RS}.
A discussão sobre os problemas, incluindo uma possível solução, é feita na 
seção \ref{ss:solucoes-compilador}.

\subsection{Macros}
\label{ss:macros-resultados}

A implementação do sistema de macros neste trabalho é, na verdade,
completamente diferente da descrita pelo \acs{R7RS}. No entanto, com a
estratégia utilizada no compilador seria impossível implementar o sistema de
macros descrito pelo \acs{R7RS}. Um sistema de macros alternativo foi
implementado que, ainda assim, possui alguns problemas caso se deseje utilizar
este interpretador em projetos reais. Os problemas no sistema de macros
implementado e as mudanças necessárias para criar um sistema de macros
compatível com o \acs{R7RS} são descritos na seção \ref{ss:solucoes-macros}.

\subsection{Módulos}
\label{ss:modulos-resultados}

O sistema de módulos descrito por \acs{R7RS} não foi implementado, novamente
por não ser possível implementá-lo sobre o compilador utilizado. Para a 
utilização do interpretador como ambiente interativo pouco se nota esta
deficiência. No entanto, caso se deseje utilizar o interpretador para a
execução de um programa real, a falta do sistema de módulos cria diversos
empecilhos para a organização do código.

Utilizando as modificações descritas na seção \ref{ss:solucoes-compilador},
no entanto, é possível implementar um sistema de módulos no interpretador
sem muitos problemas. É apenas necessário que o compilador inicialize o
contexto estático de cada módulo como um contexto vazio, incluindo neste
os vínculos necessários de acordo com quais módulos foram importados e
quais símbolos estes módulos exportam.

\subsection{Performance e memória}
\label{ss:performance-resultados}

Praticamente todos os algoritmos utilizados nesta implementação podem ter
sua performance melhorada. Estruturas de dados melhores podem ser utilizadas
para melhorar o custo de acesso a variáveis e macros. 

O consumo de memória também é muito acima do necessário. Devido à escolha de
diminuir a complexidade na gerência de memória, as estruturas utilizadas
utilizam mais memória que o necessário em uma parcela significativa dos valores
armazenados. O algoritmo de gerência de memória também não se preocupa em
reduzir fragmentação de memória, o que causa um consumo total de memória ainda
maior.

