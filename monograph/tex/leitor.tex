\section{Leitor}
\label{sec:leitor}

O módulo Leitor, uma parte tradicional da implementação de linguagens da
família Lisp é responsável por receber um fluxo de texto e processá-lo para
obter um fluxo de Expressões-S.

Sua primeira responsabilidade é a análise léxica da entrada. Ele é responsável
por identificar os tokens no texto e transformá-os em representações de
objetos atômicos no contexto das Expressões-S: constantes simbólicas,
constantes numéricas e outras constantes além de pontuação como parênteses e
ponto, e eliminar os comentários. 

Além da análise léxica, o Leitor constrói as estruturas
(listas) que compõem uma Expressão-S, agrupando os objetos de forma que todo o
resto do processo de compilação não precise se preocupar com como a informação
é armazenada textualmente, lidando apenas com Expressões-S.

Também é responsabilidade do Leitor identificar alguns caracteres especiais que
abreviam operações na linguagem: o apóstrofo (\texttt{'}), que abrevia a operação
\texttt{quote}; o acento grave (\texttt{`}), que abrevia a operação \texttt{quasiquote};
a vírgula (\texttt{,}), que abrevia a operação \texttt{unquote}; e o conjunto
``vírgula-arroba'' (\texttt{,@}), que abrevia a operação \texttt{unquote-splicing}.

Sua implementação é bastante simples: Utilizando o gerador de analisadores
léxicos GNU Flex[18] é criada uma função inicial capaz de traduzir a entrada
textual nos respectivos objetos atômicos e pontuação reconhecidos pela
estrutura das Expressões-S. Neste ponto, a análise léxica é concluída. 

Uma segunda função, que é exposta para os demais módulos, é responsável por
coordenar as chamadas à função gerada pelo GNU Flex e criar as estruturas
recursivas de uma Expressão-S, as listas e sublistas, além de substituir as 
abreviações pelas suas operações correspondentes.

Todas as estruturas criadas são alocadas com a ajuda do Sistema de Gerência de
Memória e, entre uma alocação e outra, são protegidas usando a pilha de
proteção de memória, para evitar que uma chamada à coleta de lixo durante a
execução de uma chamada ao Leitor descarte memória que foi recentemente alocada
mas ainda não está em uso pelo programa do ponto de vista da Máquina Virtual.




