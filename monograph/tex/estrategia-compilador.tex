\section{Compilador}
\label{sec:compilador}

A estratégia utilizada para o compilador é derivada de um compilador utilizado como exemplo por Kent Dybvig em sua tese de doutorado[19]. Este compilador, embora simples, sofre de algumas deficiências que serão discutidas adiante.

Durante a etapa de compilação traduzimos uma Expressão-S representando código Scheme para uma Expressão-S representando a implementação deste código nas instruções utilizadas pela máquina virtual. Observe que, como a compilação se dá em termos de Expressões-S como entrada, nada impede que um usuário forneça ao compilador uma Expressão-S que não é diretamente derivada da representação textual do programa. Esta informação será utilizada quando discutirmos o mecanismo de tradução de Macros.

Neste momento vamos abstrair o funcionamento do mecanismo de macros, dizendo apenas que antes de a compilação propriamente dita ter início, o Compilador invoca o sistema de macros para processar a Expressão-S recebida como entrada de forma a transformá-la em uma Expressão-S representando uma das poucas operações primitivas das quais o compilador tem conhecimento. Este fato simplifica muito a etapa de compilação, permitindo ao compilador se preocupar apenas com um número reduzido das estruturas sintáticas definidas no R7RS-small.

A estratégia de implementação utilizada é a mais ingênua possível. Com efeito, esta implementação não mantém informações de análise estática relacionadas a escopo, o que a faz incapaz de representar o mecanismo de macros definido pela linguagem Scheme. A implementação de um mecanismo alternativo de macros procedurais, compatível com os mecanismos tradicionais de outras linguagens da família Lisp, além de uma descrição das mudanças necessárias para se implementar o mecanismo de macros descrito por Scheme, são discutidos posteriormente na seção \ref{sec:macros}.

Dez tipos de expressão são reconhecidos pelo compilador: Referência a variável, constantes,  expressões literais, criação de procedimentos, condicionais, atribuições, definições, definições de macro, obtenção de continuação e aplicação de função. A partir destes tipos primitivos de expressão, com a ajuda do mecanismo de macros, todas as outras estruturas sintáticas da linguagem são construídas, simplificando significativamente o projeto do compilador.

O processo de compilação inicia-se com dois parâmetros: a expressão a ser compilada e a próxima etapa da computação que está sendo efetuada. Quando estamos realizando a compilação de uma expressão no escopo mais alto, esta próxima etapa normalmente é um pedaço de código pré compilado correspondendo à instrução Halt da máquina virtual. Esta expressão representando a próxima etapa da computação pode então ser utilizada para compor o parâmetro das instruções da máquina virtual que indica qual a próxima instrução a ser executada.

% TODO [7]: Terminar esta seção.



